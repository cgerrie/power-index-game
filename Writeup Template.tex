\documentclass[12pt]{article}

\usepackage{amsthm}
\usepackage{amsmath}
\usepackage{amssymb}
\usepackage{fullpage}
\usepackage{mathtools}
\usepackage{tikz}
\usetikzlibrary{arrows}
\DeclarePairedDelimiter\ceil{\lceil}{\rceil}
\DeclarePairedDelimiter\floor{\lfloor}{\rfloor}
\pagestyle{empty}

\newtheorem{theorem}{Theorem}[section]

\newtheorem{lemma}{Lemma}[section]

\newtheorem{proposition}{Proposition}[section]

\newtheorem{corollary}{Corollary}[section]

\theoremstyle{definition}
\newtheorem{definition}{Definition}[section]
 
\theoremstyle{remark}
\newtheorem{remark}{Remark}[section]

\theoremstyle{remark}
\newtheorem{example}{Example}[section]

\begin{document}

%TITLE PAGE%

\tableofcontents

\section{Introduction} \label{IntroSec}

\subsection{The Power Index Game} \label{PIgame}

\subsection{Prior Work and Variations} \label{PriorWork}

\subsection{Improvements on Prior Results} \label{Improvements}

\section{The Power Index Game on a Torus} \label{Main Chapter}

\subsection{Simplifications}

\subsection{Examples}

\subsection{Special Shapes}

%huge section
\subsection{Long Term Behaviour}

\subsubsection{Threshold for Homogeneous Stability}

\subsubsection{Behaviour of Power}

%Show why 3-3 not as important via entropy
\subsubsection{1-2 and 2-2 Proportions}

\subsubsection{Stability Over Time}

\section{Conclusion}

\subsection{Open Problems}

\end{document}
